\documentclass[]{UCD_CS_FYP_Report}
\usepackage{graphicx}
\usepackage{hyperref}

%%%%%%%%%%%%%%%%%%%%%%
%%% Input project details

\def\studentname{Luis Marron (23202882)} % Edit with your name
\def\projecttitle{{\linespread{4.5}\selectfont COMP47910 Secure Software Engineering}} % Edit with you project title
\def\supervisorname{Dr. Liliana Pasquale} % Edit with your supervisor name


\begin{document}

\maketitle

%%%%%%%%%%%%%%%%%%%%%%
%%% Table of Content

\tableofcontents\pdfbookmark[0]{Table of Contents}{toc}\newpage
\newpage

%%%%%%%%%%%%%%%%%%%%%%
%%% Penetration Testing Report


\chapter{A05:2021 Security Misconfiguration}

\section{CWE-693: Protection Mechanism Failure}

\textbf{Description}: The BookShop application does not implement Content Security Policy (CSP) headers, leaving it vulnerable to Cross-Site Scripting (XSS) and data injection attacks. CSP is a critical security layer that helps detect and mitigate various types of attacks including data theft, site defacement, and malware distribution by declaring approved sources of content that browsers should be allowed to load.

\textbf{CWE Explanation}: CWE-693 occurs when a protection mechanism (in this case, CSP headers) is not implemented or is improperly configured, allowing attackers to bypass security controls and execute malicious content.

\textbf{Severity}: High

\textbf{Risk Level}: 
\begin{itemize}
    \item \textbf{Probability}: High - CSP headers are completely absent, making XSS attacks trivial to execute
    \item \textbf{Impact}: High - Can lead to data theft, session hijacking, and malware distribution
    \item \textbf{Overall Risk}: High - Critical security control missing with high exploitation potential
\end{itemize}

\textbf{Location}: 
\begin{itemize}
    \item Frontend: \texttt{http://localhost:3000}
    \item Backend: \texttt{http://localhost:8080}
\end{itemize}

\textbf{Discovery Method}: 
\begin{itemize}
    \item \textbf{Tool Used}: OWASP ZAP (Zed Attack Proxy) Active Scan
    \item \textbf{Scan Configuration}: Default active scan policies enabled
    \item \textbf{Target}: \texttt{http://localhost:3000} and \texttt{http://localhost:8080}
    \item \textbf{Detection}: ZAP identified missing Content-Security-Policy headers in HTTP responses
\end{itemize}

\textbf{Source Code Location}: 
\begin{itemize}
    \item Frontend: \texttt{frontend/bookshop-frontend/public/index.html} (lines 6-12) - Missing CSP meta tag in HTML head section
    \item Backend: No security configuration class exists in \texttt{src/main/java/com/example/bookshop/config/} directory
    \item Application Properties: \texttt{src/main/resources/application.properties} - No CSP-related configuration
\end{itemize}

\textbf{Exploitation}:
\begin{itemize}
    \item An attacker can inject malicious JavaScript code through user input fields (e.g., book titles, descriptions) that will execute in the context of the application
    \item The absence of CSP headers allows any script source to be executed, including inline scripts and external domains
    \item Example Attack Vector (Intercepted in ZAP):
    \begin{verbatim}
    POST /admin/book HTTP/1.1
    Host: localhost:8080
    Content-Type: application/json
    Cookie: JSESSIONID=xyz789
    
    {
        "title": "<script>alert('XSS')</script>",
        "author": "Author",
        "price": 29.99,
        "copies": 10
    }
    \end{verbatim}
    \item The malicious script would execute when an admin views the book list, potentially stealing session cookies or performing unauthorized actions
    \item ZAP's Active Scan and manual testing confirmed the absence of CSP headers in HTTP responses
\end{itemize}

% TODO: INSERT SCREENSHOT - ZAP Active Scan Results showing CSP Header Not Set vulnerability
% Screenshot should show: ZAP Alerts tab with "Content Security Policy (CSP) Header Not Set" alert
% File: screenshots/zap_csp_missing.png
% Caption: "ZAP Active Scan Results - Missing CSP Headers Detected"

% TODO: INSERT SCREENSHOT - HTTP Response Headers showing absence of CSP
% Screenshot should show: ZAP Response tab with HTTP headers (no Content-Security-Policy header)
% File: screenshots/http_response_no_csp.png
% Caption: "HTTP Response Headers - Missing Content-Security-Policy Header"

\textbf{Impact}:
\begin{itemize}
    \item Enables Cross-Site Scripting (XSS) attacks, allowing execution of arbitrary JavaScript code
    \item Facilitates data theft through session hijacking and cookie stealing
    \item Enables site defacement and distribution of malware to users
    \item Compromises user privacy and application integrity
    \item May lead to regulatory violations (GDPR, CCPA) due to data exposure
\end{itemize}

\textbf{WebGoat Practice}:
\begin{itemize}
    \item Replicate in WebGoat's \textbf{Cross-Site Scripting (XSS)} lessons
    \item Practice XSS attacks in \textbf{Reflected XSS} and \textbf{Stored XSS} modules
    \item Use ZAP to intercept requests and inject malicious scripts
    \item Verify the absence of CSP headers using browser developer tools or ZAP's Response tab
    \item Practice bypassing CSP restrictions (when implemented) in WebGoat's \textbf{Client Side} > \textbf{Cross-Site Scripting} lessons
\end{itemize}

\textbf{Remediation}:
\begin{enumerate}
    \item \textbf{Backend Fix (Spring Boot)}:
    \begin{itemize}
        \item Create a security configuration class:
        \begin{verbatim}
        // src/main/java/com/example/bookshop/config/SecurityConfig.java
        @Configuration
        @EnableWebSecurity
        public class SecurityConfig {
            @Bean
            public SecurityFilterChain filterChain(HttpSecurity http) throws Exception {
                http.headers(headers -> headers
                    .contentSecurityPolicy(csp -> csp
                        .policyDirectives("default-src 'self'; " +
                            "script-src 'self' 'unsafe-inline'; " +
                            "style-src 'self' 'unsafe-inline'; " +
                            "img-src 'self' data:; " +
                            "connect-src 'self' http://localhost:3000; " +
                            "frame-ancestors 'none';")
                    )
                );
                return http.build();
            }
        }
        \end{verbatim}
        
        \item Add Spring Security dependency to \texttt{pom.xml}:
        \begin{verbatim}
        <dependency>
            <groupId>org.springframework.boot</groupId>
            <artifactId>spring-boot-starter-security</artifactId>
        </dependency>
        \end{verbatim}
    \end{itemize}
    
    \item \textbf{Frontend Fix (React)}:
    \begin{itemize}
        \item Add CSP meta tag to \texttt{frontend/bookshop-frontend/public/index.html}:
        \begin{verbatim}
        <meta http-equiv="Content-Security-Policy" 
              content="default-src 'self'; 
                       script-src 'self' 'unsafe-inline'; 
                       style-src 'self' 'unsafe-inline'; 
                       connect-src 'self' http://localhost:8080;" />
        \end{verbatim}
    \end{itemize}
    
    \item \textbf{Testing the Fix}:
    \begin{itemize}
        \item Use ZAP to verify CSP headers are present in HTTP responses
        \item Test XSS payloads are blocked by CSP policies
        \item Use browser developer tools to check CSP header implementation
    \end{itemize}
\end{enumerate}

\section{CWE-1021: Improper Restriction of Rendered UI Layers or Frames}

\textbf{Description}: The BookShop application does not implement anti-clickjacking headers, leaving it vulnerable to clickjacking attacks. The response does not protect against 'ClickJacking' attacks and should include either Content-Security-Policy with 'frame-ancestors' directive or X-Frame-Options header to prevent malicious sites from embedding the application in iframes.

\textbf{CWE Explanation}: CWE-1021 occurs when the application fails to properly restrict how its UI layers or frames can be rendered, allowing attackers to overlay malicious content over legitimate application interfaces to trick users into performing unintended actions.

\textbf{Severity}: Medium

\textbf{Risk Level}: 
\begin{itemize}
    \item \textbf{Probability}: Medium - Clickjacking attacks require user interaction but are relatively easy to execute
    \item \textbf{Impact}: Medium - Can lead to unauthorized actions, data theft, and user deception
    \item \textbf{Overall Risk}: Medium - Missing security header with moderate exploitation complexity
\end{itemize}

\textbf{Location}: 
\begin{itemize}
    \item Frontend: \texttt{http://localhost:3000}
    \item Backend: \texttt{http://localhost:8080}
\end{itemize}

\textbf{Discovery Method}: 
\begin{itemize}
    \item \textbf{Tool Used}: OWASP ZAP (Zed Attack Proxy) Active Scan
    \item \textbf{Scan Configuration}: Default active scan policies enabled
    \item \textbf{Target}: \texttt{http://localhost:3000} and \texttt{http://localhost:8080}
    \item \textbf{Detection}: ZAP identified missing X-Frame-Options and frame-ancestors headers in HTTP responses
\end{itemize}

\textbf{Source Code Location}: 
\begin{itemize}
    \item Frontend: \texttt{frontend/bookshop-frontend/public/index.html} (lines 6-12) - Missing X-Frame-Options meta tag
    \item Backend: No security configuration class exists in \texttt{src/main/java/com/example/bookshop/config/} directory
    \item Application Properties: \texttt{src/main/resources/application.properties} - No frame-ancestors configuration
\end{itemize}

\textbf{Exploitation}:
\begin{itemize}
    \item An attacker can create a malicious website that embeds the BookShop application in a transparent iframe
    \item The malicious site overlays invisible buttons or forms over the legitimate application interface
    \item Example Attack Vector (Intercepted in ZAP):
    \begin{verbatim}
    <!DOCTYPE html>
    <html>
    <head><title>Malicious Site</title></head>
    <body>
        <div style="position: relative; opacity: 0.1;">
            <iframe src="http://localhost:3000" 
                    style="width: 100%; height: 100vh; border: none;">
            </iframe>
        </div>
        <button style="position: absolute; top: 50%; left: 50%; 
                      z-index: 1000; opacity: 0.9;">
            Click here to win!
        </button>
    </body>
    </html>
    \end{verbatim}
    \item Users believe they're clicking on legitimate buttons but actually perform actions on the embedded BookShop application
    \item ZAP's Active Scan detected the missing anti-clickjacking headers in HTTP responses
\end{itemize}

% TODO: INSERT SCREENSHOT - ZAP Active Scan Results showing Missing Anti-clickjacking Header
% Screenshot should show: ZAP Alerts tab with "Missing Anti-clickjacking Header" alert
% File: screenshots/zap_clickjacking_missing.png
% Caption: "ZAP Active Scan Results - Missing Anti-clickjacking Headers Detected"

% TODO: INSERT SCREENSHOT - HTTP Response Headers showing missing X-Frame-Options
% Screenshot should show: ZAP Response tab with HTTP headers (no X-Frame-Options header)
% File: screenshots/http_response_no_xframe.png
% Caption: "HTTP Response Headers - Missing X-Frame-Options Header"

\textbf{Impact}:
\begin{itemize}
    \item Enables clickjacking attacks where users perform unintended actions
    \item Can lead to unauthorized purchases, account modifications, or data exposure
    \item Compromises user trust and application integrity
    \item May result in financial losses or privacy violations
\end{itemize}

\textbf{WebGoat Practice}:
\begin{itemize}
    \item Replicate in WebGoat's \textbf{Client Side} > \textbf{Cross-Site Scripting} lessons
    \item Practice iframe-based attacks in \textbf{Cross-Site Request Forgery (CSRF)} modules
    \item Use ZAP to test frame-ancestors and X-Frame-Options headers
    \item Practice creating malicious iframe overlays in WebGoat's clickjacking exercises
    \item Test browser security policies using developer tools
\end{itemize}

\textbf{Remediation}:
\begin{enumerate}
    \item \textbf{Option 1: Content Security Policy (CSP) frame-ancestors}:
    \begin{itemize}
        \item Add frame-ancestors directive to CSP header in \texttt{SecurityConfig.java}:
        \begin{verbatim}
        .contentSecurityPolicy(csp -> csp
            .policyDirectives("default-src 'self'; " +
                "script-src 'self' 'unsafe-inline'; " +
                "style-src 'self' 'unsafe-inline'; " +
                "img-src 'self' data:; " +
                "connect-src 'self' http://localhost:3000; " +
                "frame-ancestors 'none';")
        )
        \end{verbatim}
        
        \item For same-origin framing only, use: \texttt{frame-ancestors 'self';}
    \end{itemize}
    
    \item \textbf{Option 2: X-Frame-Options Header}:
    \begin{itemize}
        \item Add X-Frame-Options to security configuration:
        \begin{verbatim}
        http.headers(headers -> headers
            .frameOptions().deny()  // or .sameOrigin()
        )
        \end{verbatim}
        
        \item Use \texttt{.deny()} to prevent all framing or \texttt{.sameOrigin()} for same-origin only
    \end{itemize}
    
    \item \textbf{Frontend Fix (React)}:
    \begin{itemize}
        \item Add X-Frame-Options meta tag to \texttt{frontend/bookshop-frontend/public/index.html}:
        \begin{verbatim}
        <meta http-equiv="X-Frame-Options" content="DENY" />
        \end{verbatim}
        
        \item Or for same-origin only: \texttt{content="SAMEORIGIN"}
    \end{itemize}
    
    \item \textbf{Testing the Fix}:
    \begin{itemize}
        \item Use ZAP to verify X-Frame-Options or frame-ancestors headers are present
        \item Test iframe embedding is properly blocked in browser developer tools
        \item Verify malicious iframe overlays are prevented
        \item Check that legitimate same-origin framing still works (if using SAMEORIGIN)
    \end{itemize}
\end{enumerate}

\chapter{A07:2021 Identification and Authentication Failures}

\section{CWE-264: Permissions, Privileges, and Access Controls}

\textbf{Description}: The BookShop application has a Cross-Origin Resource Sharing (CORS) misconfiguration that permits cross-domain read requests from arbitrary third-party domains to unauthenticated APIs. While web browser implementations do not permit arbitrary third parties to read responses from authenticated APIs, this misconfiguration could still be exploited by attackers to access sensitive data through unauthenticated endpoints.

\textbf{CWE Explanation}: CWE-264 occurs when the application fails to properly restrict cross-origin requests, allowing unauthorized domains to access resources and potentially sensitive data.

\textbf{Severity}: Medium

\textbf{Risk Level}: 
\begin{itemize}
    \item \textbf{Probability}: Medium - CORS is configured but allows specific origins, reducing but not eliminating risk
    \item \textbf{Impact}: Medium - Can lead to unauthorized data access but limited by browser security policies
    \item \textbf{Overall Risk}: Medium - Misconfiguration exists but with some mitigating factors
\end{itemize}

\textbf{Location}: 
\begin{itemize}
    \item Frontend: \texttt{http://localhost:3000}
    \item Backend: \texttt{http://localhost:8080}
\end{itemize}

\textbf{Discovery Method}: 
\begin{itemize}
    \item \textbf{Tool Used}: OWASP ZAP (Zed Attack Proxy) Active Scan
    \item \textbf{Scan Configuration}: Default active scan policies enabled
    \item \textbf{Target}: \texttt{http://localhost:3000} and \texttt{http://localhost:8080}
    \item \textbf{Detection}: ZAP identified CORS misconfiguration allowing cross-domain requests from arbitrary origins
\end{itemize}

\textbf{Source Code Location}: 
\begin{itemize}
    \item \texttt{src/main/java/com/example/bookshop/controller/BookControl.java} (line 10) - \texttt{@CrossOrigin(origins = "http://localhost:3000", allowCredentials = "true")}
    \item \texttt{src/main/java/com/example/bookshop/controller/AdminControl.java} (line 13) - \texttt{@CrossOrigin(origins = "http://localhost:3000", allowCredentials = "true")}
    \item \texttt{src/main/java/com/example/bookshop/controller/AuthControl.java} (line 10) - \texttt{@CrossOrigin(origins = "http://localhost:3000", allowCredentials = "true")}
    \item \texttt{src/main/java/com/example/bookshop/controller/CartControl.java} (line 13) - \texttt{@CrossOrigin(origins = "http://localhost:3000", allowCredentials = "true")}
\end{itemize}

\textbf{Exploitation}:
\begin{itemize}
    \item An attacker can create a malicious website that makes cross-origin requests to the BookShop API endpoints
    \item The \texttt{@CrossOrigin} annotation allows requests from \texttt{http://localhost:3000} with credentials
    \item Example Attack Vector (Intercepted in ZAP):
    \begin{verbatim}
    GET /books HTTP/1.1
    Host: localhost:8080
    Origin: http://malicious-site.com
    Access-Control-Allow-Origin: http://localhost:3000
    Access-Control-Allow-Credentials: true
    \end{verbatim}
    \item The unauthenticated \texttt{/books} endpoint returns all book data, which could be accessed by malicious sites
    \item ZAP's Active Scan detected the CORS misconfiguration and flagged it as a potential security issue
\end{itemize}

% TODO: INSERT SCREENSHOT - ZAP Active Scan Results showing CORS Misconfiguration
% Screenshot should show: ZAP Alerts tab with "Cross-Domain Misconfiguration" alert
% File: screenshots/zap_cors_misconfig.png
% Caption: "ZAP Active Scan Results - CORS Misconfiguration Detected"

% TODO: INSERT SCREENSHOT - HTTP Response Headers showing CORS configuration
% Screenshot should show: ZAP Response tab with Access-Control-Allow-Origin headers
% File: screenshots/http_response_cors_headers.png
% Caption: "HTTP Response Headers - CORS Configuration Analysis"

\textbf{Impact}:
\begin{itemize}
    \item Unauthorized access to sensitive data through unauthenticated API endpoints
    \item Potential data leakage of book information, prices, and inventory details
    \item Enables cross-site request forgery (CSRF) attacks when combined with other vulnerabilities
    \item Violates the Same Origin Policy (SOP) security model
\end{itemize}

\textbf{WebGoat Practice}:
\begin{itemize}
    \item Replicate in WebGoat's \textbf{Cross-Site Request Forgery (CSRF)} lessons
    \item Practice CORS bypass techniques in \textbf{Client Side} > \textbf{Cross-Site Scripting} modules
    \item Use ZAP to intercept and modify CORS headers in requests
    \item Test cross-origin requests using browser developer tools or ZAP's Active Scan
    \item Practice implementing proper CORS policies in WebGoat's security configuration lessons
\end{itemize}

\textbf{Remediation}:
\begin{enumerate}
    \item \textbf{Option 1: Remove CORS Entirely (Recommended for Internal Apps)}:
    \begin{itemize}
        \item Remove all \texttt{@CrossOrigin} annotations from controller classes:
        \begin{itemize}
            \item \texttt{src/main/java/com/example/bookshop/controller/BookControl.java} (line 10)
            \item \texttt{src/main/java/com/example/bookshop/controller/AdminControl.java} (line 13)
            \item \texttt{src/main/java/com/example/bookshop/controller/AuthControl.java} (line 10)
            \item \texttt{src/main/java/com/example/bookshop/controller/CartControl.java} (line 13)
        \end{itemize}
        \item This allows browsers to enforce Same Origin Policy (SOP) more restrictively
    \end{itemize}
    
    \item \textbf{Option 2: Global CORS Configuration (For Multi-Domain Apps)}:
    \begin{itemize}
        \item Create a centralized CORS configuration:
        \begin{verbatim}
        // src/main/java/com/example/bookshop/config/CorsConfig.java
        @Configuration
        public class CorsConfig {
            @Bean
            public CorsConfigurationSource corsConfigurationSource() {
                CorsConfiguration configuration = new CorsConfiguration();
                configuration.setAllowedOrigins(Arrays.asList("http://localhost:3000"));
                configuration.setAllowedMethods(Arrays.asList("GET", "POST", "PUT", "DELETE"));
                configuration.setAllowedHeaders(Arrays.asList("*"));
                configuration.setAllowCredentials(true);
                
                UrlBasedCorsConfigurationSource source = new UrlBasedCorsConfigurationSource();
                source.registerCorsConfiguration("/**", configuration);
                return source;
            }
        }
        \end{verbatim}
        
        \item Remove individual \texttt{@CrossOrigin} annotations from controllers
    \end{itemize}
    
    \item \textbf{Option 3: Add Authentication to Sensitive Endpoints}:
    \begin{itemize}
        \item Implement authentication for the \texttt{/books} endpoint:
        \begin{verbatim}
        @GetMapping
        @PreAuthorize("hasRole('USER')")
        public List<Book> listBooks() {
            return bookRepo.findAll();
        }
        \end{verbatim}
        
        \item This ensures sensitive data is not available in unauthenticated manner
    \end{itemize}
    
    \item \textbf{Testing the Fix}:
    \begin{itemize}
        \item Test cross-origin requests are properly blocked or restricted
        \item Use browser developer tools to check CORS headers
        \item Verify ZAP no longer flags CORS misconfiguration
    \end{itemize}
\end{enumerate}

\chapter{Conclusions}

The BookShop application exhibits a critical \textbf{OWASP Top 10 A5: Security Misconfiguration} vulnerability (CWE-693), specifically the absence of Content Security Policy headers. This security misconfiguration enables Cross-Site Scripting attacks, data theft, and potential malware distribution. The vulnerability affects both the frontend React application and the backend Spring Boot API, requiring immediate remediation through proper CSP header implementation. Practice these vulnerabilities in WebGoat to build detection and exploitation skills before testing production applications.



\chapter{Prepared By}
\begin{itemize}
    \item \textbf{Tester}: Luis Marron
    \item \textbf{Date}: January 2025
    \item \textbf{Contact}: luis.marron@ucdconnect.ie
\end{itemize}

%%%%  BIBLIOGRAPHY 
\newpage
\begin{thebibliography}{99}
\bibitem{DAWSON:1994} Erich Gamma, Richard Helm, Ralph Johnson, and John Vlissides. \emph{ Design Patterns: Elements of Reusable Object-Oriented Software. Addison-Wesley}. 395 pages. ISBN-13: 978-0-201-63361-0. Pearson Education, 1994.
\end{thebibliography}
\label{endpage}

\end{document}

\end {article}
